\documentclass{scrartcl}

% configuration lifted from chez-sty at https://github.com/chezbgone/chez-sty
\KOMAoptions{paper=letter, fontsize=10pt, usegeometry}

% standard
\usepackage{mathtools, amssymb, amsthm}

% refs
\usepackage{xcolor}
\usepackage[pdftex, colorlinks]{hyperref}
\definecolor{urlcolor}{rgb}{0.9, 0, 0.5}
\definecolor{linkcolor}{rgb}{0, 0.2, 0.8}
\definecolor{citecolor}{rgb}{0, 0.6, 0.3}
\hypersetup{urlcolor=urlcolor, linkcolor=linkcolor, citecolor=citecolor}
\usepackage[nameinlink]{cleveref}

% layout
\usepackage[margin=1.5cm, includefoot]{geometry}
\usepackage[nodisplayskipstretch, onehalfspacing]{setspace}
% section numbers in margin
\RenewDocumentCommand{\sectionlinesformat}{m m m m}{\makebox[0pt][r]{#3}#4}

% typography
\usepackage[T1]{fontenc}
\usepackage{libertine}
\usepackage[scaled=0.83]{beramono}
\usepackage{microtype}
\usepackage[shortlabels, inline]{enumitem}
\setlist{nosep, listparindent=\parindent}

% smaller title style
\makeatletter \patchcmd{\@maketitle}{\huge}{\large}{}{} \makeatother
\setkomafont{author}{\normalsize\scshape}
\setkomafont{date}{\normalsize}
\RedeclareSectionCommand[beforeskip=0.4\baselineskip]{paragraph}

% temp config
\definecolor{todocolor}{rgb}{0.8, 0.1, 0.1}
\NewDocumentCommand{\TODO}{o}{\textcolor{todocolor}{TODO\IfNoValueF{#1}{: #1}}}

% % % % % % % % % % % % % % % % % % % % %
% actual content

\title{Schedgehog: a property-based testing library}
\author{Jason Chen \and CJ Quines \and Matthew Ho}

\begin{document}
\maketitle

\section{Overview}

In software engineering, testing code is considered to be a vital part of
delivering quality software. However, writing unit tests is extremely time
consuming, and it is easy for the programmer to miss certain cases for more
complicated functions. Therefore, it is desirable to try to automate this
processes either partially, or entirely.

We present a tool called \textbf{Schedgehog}, based on existing Haskell work
like QuickCheck \cite{quickcheck} or Hedgehog \cite{hedgehog}, that allows
users to formulate properties of programs and then automatically runs tests
based on the properties. Properties that users want to test will be specified
via a DSL embedded in Scheme; these are represented as predicates (functions
returning booleans) along with typing information of the inputs.

Test data is generated based on this typing information. These are checked
against the predicates, and in the case that a failing counterexample is found,
the counterexample is simplified via a build-in extensible rule-based system.

\iffalse
Schedgehog's generators
are generic functions associated with a type that serve two main purposes:
\begin{itemize}
  \item \emph{value generation}, which is the generation of new value of
        the type given size and seed parameters, and
  \item \emph{test shrinking}, which allows automatically simplifying failing
        counterexamples, via a built-in user-extensible rule-based system.
\end{itemize}
\fi

\section{Basic usage}

\subsection{Defining properties}
Properties are specified with the macro \verb|forall|. Properties are functions
with typed arguments that return a boolean.

\begin{verbatim}
(define prop:addition-commutativity
  (forall ((x integer) (y integer))
          (= (+ x y) (+ y x))))
\end{verbatim}

Primitive types include \verb|integer|, \verb|boolean|, and \verb|float|. A full
list of built-in primitive types are in \verb|src/generator-instances.scm|.
There are also dependent types, like \verb|linear| or \verb|range|, that take
in values as arguments.

\begin{verbatim}
(define prop:addition-bounds
  (forall ((x (range 10 30)) (y (range 10 30)))
          (< 20 (+ x y) 60)))
\end{verbatim}

These types can be combined with type constructors to produce more complex
types. Some examples of type constructors include \verb|pair| and \verb|list|.

\begin{verbatim}
(define prop:sum-linear
  (forall ((xs (list integer)) (factor integer))
          (equal? (* factor (apply + xs))
                  (apply + (map (lambda (x) (* factor x)) xs)))))
\end{verbatim}

\subsection{Checking properties}
The main interface for checking a property is with the function \verb|check|.
This runs the function against several random test cases. It then prints
the result of the test, along with a counterexample if found.

\begin{verbatim}
(check prop:addition-commutativity)
; seed: ...
; ok 100 tests
(check prop:addition-bounds)
; seed: ...
; failed after 82 tests, 0 shrinks:
; ((x 10) (y 10))
\end{verbatim}

The function \verb|check-with-config| allows checking a property with a custom
configuration (which contains the maximum number of shrinks and tests to run),
size (which determines the size of generated inputs), and seed.

\begin{verbatim}
(check-with-config
  (make-config 1000 1000)
  1000
  (make-random-state #t)
  prop:addition-commutativity)
; ok 1000 tests
\end{verbatim}

\section{Generators}\label{sec:generators}
\subsection{Overview}

The file \verb|examples/generator.scm| contains extensive examples of the
library of built-in generators.

% TODO: maybe better syntax for function parameters
A generator represents a source of values for randomly checking a property,
along with the way in which values in the type shrink.
Intuitively, it can be thought of as a structure containing two components:
\begin{itemize}
\item A generating function \verb|(size -> seed -> value)|
\item A shrinking function \verb|(value -> list value)|
\end{itemize}
Internally, a generator is actually a wrapper around just one function
\verb|(size -> seed -> lazy-tree value)|, a function returning a lazy tree of
values. This design decision will be explained in
\cref{sec:trees}.
To create a generator without dealing with lazy tree internals,
we can call library functions that abstract away these details.

% TODO: figure out the hbox here
Suppose one had a generating function \verb|gen|. The simplest way to make a
generator would be to call \verb|(make-generator-without-shrinking gen)|,
which returns a generator object using the generating function that does not
have a shrinker.
\begin{verbatim}
(define gen-1 (make-generator-without-shrinking (lambda (size seed) 1)))
\end{verbatim}

We can then use the generator with the \verb|generate| function,
which optionally receives a size and seed as parameters.
\begin{verbatim}
(print-lazy-tree (generate gen-1))
\end{verbatim}

We can register a generator with a type by calling \verb|set-generator!|. This
allows it to be called with \verb|arbitrary|, a generic function.

\begin{verbatim}
(set-generator! 'type-1 gen-1)
(print-lazy-tree (generate (arbitrary 'type-1)))
(check (forall (x type-1) (= x 1)))
\end{verbatim}

To replace the shrink function of a generator \verb|gen| with a specified shrink
function \verb|shrink|, we call \verb|(replace-shrinking shrink gen)|.

\begin{verbatim}
(define (shrink-fn value) (case value ((1) (list 0)) ((0) (list))))
(define gen-2 ((replace-shrinking shrink-fn) gen-1))
(print-lazy-tree (generate gen-2))
\end{verbatim}

We can also create parametrized generators. The function \verb|arbitrary|, when
given a list of symbols, will instead search for a handler for the first symbol,
to be called on the list of the remaining symbols as arguments, expecting a
generator returned.

\begin{verbatim}
(define (gen-constant val)
  (make-generator-without-shrinking (lambda (size seed) val)))
(set-generator! 'type-constant (lambda (val) (gen-constant val)))
(print-lazy-tree (generate (arbitrary '(type-constant 1))))
\end{verbatim}

Note that the type of generators forms a monad. The monad methods are in the
library as \verb|gen:pure| and \verb|gen:bind|; we also provide convenience
functions \verb|gen:map|, \verb|gen:apply|, \verb|gen:list|, etc.

\section{Trees}\label{sec:trees}
\subsection{Rationale}
As mentioned in \cref{sec:generators}, a generator for values of type \verb|T|,
denoted as a \verb|(generator T)|,
can be thought of as a pair of functions,
\begin{itemize}
  \item \verb|gen| for value generation
        with type \verb|(size -> seed -> T)|, and
  \item \verb|shrink| for shrinking with type \verb|(T -> list T)|.
\end{itemize}
However, representing a generator as a collection of these two functions,
referred to as \emph{manual shrinking}, has some downsides.
% TODO: cite https://www.tweag.io/blog/2021-07-21-qcheck2-integrated-shrinking/
%       and https://www.well-typed.com/blog/2019/05/integrated-shrinking/

One desired property of a generator is \emph{invariant preservation}.
Many generators represent types with some invariant,
and we want both \verb|gen| and \verb|shrink| to satisfy these invariants.
For instance if we are generating integers between 10 and 20,
\verb|gen| must only generate numbers in this range,
and \verb|shrink| must not return a number in this range.
Both functions trying to satisfy these invariants
often leads to duplicate code, which is tedious to write and
makes it more likely for discrepancies to arise.

Another desired property of a generator is
ease of composition with other functions.
If we have a \verb|(generator T)| and a function \(f\) of type \verb|(T -> S)|,
it seems like we should be able to obtain a \verb|(generator S)| just by
mapping the function over the values that \verb|gen| gives.
This is indeed true, but we cannot do the same thing for \verb|shrink|,
as it takes \verb|T|s as both inputs and outputs.
This is called parameter invariance, which makes composition difficult.

\subsection{Structure}
To solve these issues in the generator structure,
we combine the two functions needed for generation and shrinking.
To do so, whenever we generate a value,
we also generate the values that it shrinks to,
and then recursively generate the values those new values shrink to, and so on.
This results in a generator structure that is a wrapper around a single
function \verb|(size -> seed -> (tree T))|.

The type \verb|(tree T)| is a record structure containing
  a value of type \verb|T| and
  a list of children of type \verb|(list (tree T))|s.
In the context of a generator, it represents
  a generated value
  along with the list of shrunk values and their own shrink values.

To make a slight optimization, we can note that the children of the root
of the tree are not actually needed until a generated value is
a failing example of a property.
As a result, we can make the list of children lazy,
since generating the shrunk values is unnecessary until needed in shrinking.


% \section{Combinators}

% talk about combinators

% \section{Extensions}

% idk
% coarbitrary (function types)

\begin{thebibliography}{99}

\bibitem{quickcheck}
Claessen, K., \& Hughes, J. (2011). QuickCheck: a lightweight tool for random
testing of Haskell programs. \textit{ACM SigPlan Notices}, 46(4), 53-64.

\bibitem{hedgehog}
Hedgehog QA. (2019). Hedgehog will eat all your bugs. Retrieved from
\url{https://hedgehog.qa/}.

\end{thebibliography}

\end{document}
